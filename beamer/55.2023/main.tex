\documentclass{beamer}
\usetheme{default}  % You can choose a different theme

\title{GATE 2023(ST) Question -55}
\author{RAMBHA SATVIK - EE22BTECH11043}
\institute{EE2102 - IITH}
\date{\today}

\begin{document}

\begin{frame}
\titlepage
\end{frame}

\begin{frame}
\frametitle{Question}
Suppose that $U$ and $V$ are two independent and identically distributed random
variables each having probability density function
\begin{align}
f(x) = 
\begin{cases}
\lambda^{2}xe^{-\lambda x} & \text{if } x > 0\\
0 & Otherwise,
\end{cases}
\end{align}
where $\lambda > 0$. Which of the following statements is/are true?
\begin{enumerate}
\item The distribution of $U-V$ is symmetric about 0
\item The distribution of $UV$ does not depend on $\lambda$
\item The distribution of $\frac{U}{V}$ does not depend on $\lambda$
\item The distribution of $\frac{U}{V}$ is symmetric about 1
\end{enumerate}
\end{frame}

\begin{frame}
\frametitle{Solution}
A continuous random variable X is said to have a gamma distribution with parameters $\alpha>0$ and $\lambda>0$, shown as $X \sim Gamma(\alpha,\lambda)$
, if its PDF is given by
\begin{align}
p_{X}(x) &= 
\begin{cases}
\frac{\lambda^{\alpha}x^{\alpha-1}e^{-\lambda x}}{\Gamma(\alpha)} & \text{if } x > 0\\
0 & Otherwise,
\end{cases}\\
E(X)&=\frac{\alpha}{\lambda}
\end{align}
It is clear that both $U$ and $V$ are of Gamma distribution i.e, $U,V \sim Gamma(2,\lambda)$. Both U and V are symmetric ditributions about mean,
\begin{align}
E(U)=E(V)=\frac{2}{\lambda} \label{eq:55_2023_3}
\end{align}
\end{frame}
\begin{frame}
\frametitle{Option 1}
As $U$ and $V$ are of same distribution, 
\begin{align}
E(U-V) = E(U) - E(V) = 0
\end{align}
Hence, the distribution of U-V is symmetric about 0.\\
Option 1 is correct.
\end{frame}
\begin{frame}
\frametitle{Option 2}
As U and V are independent and identically distributed,
\begin{align}
E(U.V) &= E(U)\times E(V) \\
&= \frac{4}{\lambda^{2}}
\end{align}
The distribution of U.V depends on $\lambda$.\\
Option 2 is incorrect
\end{frame}
\begin{frame}[allowframebreaks]
\frametitle{Option 3}
When $X \sim Gamma(\alpha,\lambda)$,
\begin{align}
E(X) = \frac{\alpha}{\lambda}
\end{align}
In that case, $X^{-1}\sim inv.Gamma(\alpha,\beta)$,
\begin{align}
E(X^{-1}) = \frac{\beta}{\alpha-1} \label{eq:55_2023_1}
\end{align}
Using \eqref{eq:55_2023_1} we can evaluate the distribution for $\frac{1}{V}$.
\begin{align}
E(V^{-1}) = \frac{1}{\lambda}\label{eq:55_2023_2}
\end{align}
Using \eqref{eq:55_2023_2} and \eqref{eq:55_2023_3}, 
\begin{align}
E\brak{\frac{U}{V}} &= E(U).E(V^{-1})\\
&= \frac{2}{\lambda}.\lambda \\ 
&= 2 \label{eq:55_2023_4}
\end{align}
$E\brak{\frac{U}{V}}$ does not depend on $\lambda$.\\
Option 3 is correct.
\end{frame}
\begin{frame}
\frametitle{Option 4}
From \eqref{eq:55_2023_4}, distribution of $\frac{U}{V}$ is symmetric about 2 (since it is its mean) and not 1.\\
Option 4 is not correct.
\end{frame}
\end{document}

