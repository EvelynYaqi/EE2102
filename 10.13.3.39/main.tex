\let\negmedspace\undefined
\let\negthickspace\undefined
\documentclass[journal,12pt,twocolumn]{IEEEtran}
\usepackage{cite}
\usepackage{amsmath,amssymb,amsfonts,amsthm}
\usepackage{algorithmic}
\usepackage{graphicx}
\usepackage{textcomp}
\usepackage{xcolor}
\usepackage{txfonts}
\usepackage{listings}
\usepackage{enumitem}
\usepackage{mathtools}
\usepackage{gensymb}
\usepackage{comment}
\usepackage[breaklinks=true]{hyperref}
\usepackage{tkz-euclide} 
\usepackage{listings}
\usepackage{gvv}                                        
\def\inputGnumericTable{}                                 
\usepackage[latin1]{inputenc}                                
\usepackage{color}                                            
\usepackage{array}                                            
\usepackage{longtable}                                       
\usepackage{calc}                                             
\usepackage{multirow}                                         
\usepackage{hhline}                                           
\usepackage{ifthen}                                           
\usepackage{lscape}

\newtheorem{theorem}{Theorem}[section]
\newtheorem{problem}{Problem}
\newtheorem{proposition}{Proposition}[section]
\newtheorem{lemma}{Lemma}[section]
\newtheorem{corollary}[theorem]{Corollary}
\newtheorem{example}{Example}[section]
\newtheorem{definition}[problem]{Definition}
\newcommand{\BEQA}{\begin{eqnarray}}
\newcommand{\EEQA}{\end{eqnarray}}
\newcommand{\define}{\stackrel{\triangle}{=}}
\theoremstyle{remark}
\newtheorem{rem}{Remark}
\begin{document}

\bibliographystyle{IEEEtran}
\vspace{3cm}

\title{Exemplar - 10.13.3.39}
\author{EE22BTECH11043 - Rambha Satvik$^{*}$% <-this % stops a space
}
\maketitle
\newpage
\bigskip

\renewcommand{\thefigure}{\theenumi}
\renewcommand{\thetable}{\theenumi}

A die has its face marked 0,1,1,1,6,6. Two such dice are thrown together and their score is recorded.
\begin{enumerate}
	\item How many different scores are possible ?
	\item What is the probability of getting a total 7 ?  
\end{enumerate}
\\
\solution
Let the random variables be defined as:
\begin{table}[!ht]
	\input{/home/lancelot/Latex/EE2102/10.13.3.39/tables/table.tex}
\end{table}
\begin{enumerate}
\item \textbf{Possible outcomes:} The following data can be interpreted from the data given in the question,
	\begin{align}
		p_X(k) &= 
		\begin{cases}
			\frac{1}{6} & \text{if } k = 0 \\
			\frac{1}{2} & \text{if } k = 1 \\
			\frac{1}{3} & \text{if } k = 6 \\
			0 & \text{Otherwise}
		\end{cases}
		\\p_Y(k) &= 
		\begin{cases}
			\frac{1}{6} & \text{if } k = 0 \\
			\frac{1}{2} & \text{if } k = 1 \\
			\frac{1}{3} & \text{if } k = 6 \\
			0 & \text{Otherwise}
		\end{cases}\\
	\end{align}
	The probability mass function for the case where total score of both the dice is 'k' is,
	\begin{align}
		p_{X+Y}(k) &= \pr{X+Y=k}\\
			&= \pr{X=k-Y}\\
			&= E\brak{p_X(k-Y)}\\
			&= \sum_{i=0}^{6}\brak{p_X(k-i)}\brak{p_Y(i)}
	\end{align}
	The possible outcomes: 0,1,2,6,7\&12
\begin{figure}[h!]
        \includegraphics[width=\columnwidth]{/home/lancelot/Latex/EE2102/10.13.3.39/figs/plot.png}
        \caption{Sketch of Probability Mass Function for Sum} 
\end{figure}
\item \textbf{Probability of getting a 7 :} 
	\begin{align}
		p_{X+Y}(7) &= \sum_{i=0}^{6}\brak{p_X(7-i)}\brak{p_Y(i)}\\
			&= p_X(6)p_Y(1) + p_X(1)p_Y(6)\\
			&= \frac{1}{3}\times\frac{1}{2} + \frac{1}{2}\times\frac{1}{3}\\
			&= \frac{1}{3}
	\end{align}
\end{enumerate}
\end{document}
